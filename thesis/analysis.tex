\clearpage
\section{Analysis of market-available solutions}

\subsection{HP 7470A}

Although no longer popular, plotters used to be a commonly found piece of office
equipment. Designed to connect to a personal computer, they were used to plot
graphics before inkjet printers became available. One such plotter is the
Hewlett-Packard 7470A, introduced in 1982.

The 7470A used a felt-tip pen to draw monochrome graphics on A4 paper.
Communication was established over an RS-232 serial connection, and it used a
proprietary language called HP-GL to encode vector data. Software for the device
was often written in BASIC. Its task was to generate HP-GL instructions from
input data according to hard-coded algorithms \cite{7470a}.

The HP-GL language offered capabilities similar to those of G-code, while also
defining a protocol for serial communication required to obtain feedback from
the device. A handshake procedure was employed to govern the transmission of
data, so as not to overflow the machine's serial buffer.

Functionally, the machine was capable of:
\begin{itemize}
    \item Moving freely in the horizontal plane, raising and lowering the
    pen
    \item Drawing lines and arcs
    \item Drawing text using multiple fonts
    \item Scaling and translating output coordinates
    \item Switching between unit systems
    \item Switching between absolute and relative coordinates
    \item Moving at different speeds
\end{itemize}
The above list fully encompasses the features of standard G-code, but also it
makes the machine useful without the need for dedicated control software. The
7470A could be hand-operated from a simple serial terminal.

\subsection{LinuxCNC}

A more modern CNC solution is LinuxCNC, a software suite designed to bypass the
control firmware of a CNC machine and to interact with the machine's hardware
directly. By using a real-time Linux kernel and communicating over the
high-speed PCI bus, the software makes it possible to control stepper motors
with a personal computer \cite{linuxcnc}.

Not being linked to any manufacturer gives LinuxCNC the freedom to accept any
data format and to directly execute arbitrary toolpath generation algorithms.
Nonetheless, the software implements its own G-code parser with a wide range of
functionality equivalent to that of commercial manufacturers. Features include:
\begin{itemize}
    \item Linear and rotational axes of motion
    \item Additional toolpath curves
    \item Rotation and translation of the coordinate system
    \item Compensation for tool size during movement
    \item Canned cycles for lathes and drills
    \item Multiple tools
\end{itemize}

Being a purely software solution, the system allows the user to inspect the
CNC machine state in great detail. The user may step through a CNC program and
view generated toolpaths before and after they are executed. Current parameters
of the machine, including its configuration, position and memory state, are
displayed on-screen.
