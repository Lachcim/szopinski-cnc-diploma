\documentclass[
    bindingoffset=5mm,
    footnoteindent=3mm,
    hyphenation=true
]{template/wut-thesis}

\facultyeiti
\EngineerThesis
\langeng

\graphicspath{{img}}
\addbibresource{bibliography.bib}

\begin{document}

\instytut{Computer Science}
\kierunek{Computer Science}
\specjalnosc{Computer Systems and Networks}
\title{
    General-purpose plotter with control software
}
\poltitle{
    Ploter ogólnego zastosowania z oprogramowaniem sterującym
}
\author{Michał Szopiński}
\album{300182}
\promotor{mgr inż. Maciej Urbański}
\date{2022}
\maketitle

% english abstract
\cleardoublepage
\abstract \lipsum[1-3]
\keywords XXX, XXX, XXX

% polish abstract
\clearpage
\secondabstract \kant[1-3]
\secondkeywords XXX, XXX, XXX

\pagestyle{plain}

\cleardoublepage
\tableofcontents

\cleardoublepage
\pagestyle{headings}

% thesis sections
\clearpage % Rozdziały zaczynamy od nowej strony.
\section{Praefatio}

% Akapit z cytatem
\lipsum[1] \cite{goossens93}

% Przykładowy obrazek
\begin{figure}[!h]
    % Wyrównanie obrazka, szerokość i plik
    % Zamiast width można też użyć height, etc.
    \centering hello
    % Podpis umieszczamy pod obrazkiem
    % znacznik \caption służy również do wygenerowania numeru obrazka
    \caption{Tradycyjne godło Politechniki Warszawskiej}
    % \label pozwala odwołać się do obrazka w innych miejscach za pomocą \ref
    % odwołanie \ref renderuje się jako numer obrazka,
    % dlatego zawsze najpierw używaj \caption a potem \label
    \label{fig:tradycyjne-logo-pw}
\end{figure}

% \ref wyrenderuje się jako 'Reference to image 1.1'
\lipsum[2] Reference to image \ref{fig:tradycyjne-logo-pw}.

% Lista punktowana
% Parametr label ustawia symbol punktora
\begin{itemize}
    \item Item 1:
    \begin{itemize}[label=---]
        \item item 1.1;
        \item item 1.2;
    \end{itemize}
    \item Item 2;
    \item Item 3.
\end{itemize}

\lipsum[3]

% Lista numerowana w formacie 1.a).ii
% Tutaj również można stosować \label
\begin{enumerate}
    \item Item 1:
    \begin{enumerate}
        \item item 1.1;
        \item item 1.2:
        \begin{enumerate}
            \item item 1.2.1;
            \item item 1.2.2;
        \end{enumerate}
        \item item 1.3;
    \end{enumerate}
    \item Item 2;
    \item Item 3.
\end{enumerate}

% Przypis dolny \footnote
\lipsum[4] Lorem ipsum dolor sit amet\footnote{Lorem ipsum dolor sit amet, consectetur adipiscing elit, sed do eiusmod tempor incididunt ut labore et dolore magna aliqua. Ut enim ad minim veniam, quis nostrud exercitation ullamco laboris nisi ut aliquip ex ea commodo consequat.}, consectetur adipiscing elit.

% Przykładowa tabela: wyśrodkowana i renderowana
% w miejscu wstawienia: !h = !h[ere]
% Domyślnie tabele trafiają na górę strony
\begin{table}[!h] \centering
    % Podpis tabeli umieszczamy od góry
    \caption{Przykładowa tabela.}
    \label{tab:tabela1}

    % Tabela z trzema kolumnami:
    % dwie wyrównanie do środka [c], a ostatnia do prawej [r]
    % szerokość kolumn automatyczna (równa szerokości tekstu)
    \begin{tabular}{| c | c | r |} \hline
        Kolumna 1       & Kolumna 2 & Liczba \\ \hline\hline
        cell1           & cell2     & 60     \\ \hline
        cell4           & cell5     & 43     \\ \hline
        cell7           & cell8     & 20,45  \\ \hline
        % Komórka o szerokości dwóch kolumn, wyrównana do prawej
        % Przypisy dolne w tabelach wstawiamy przez \tablefootnote
        \multicolumn{2}{|r|}{Suma\tablefootnote{Table footnote.}} & 123,45 \\ \hline
    \end{tabular}

\end{table}

Lorem ipsum dolor sit amet.

\clearpage % Rozdziały zaczynamy od nowej strony.
\section{De Finibus Bonorum et Malorum}

% Równanie typu 'inline':
\lipsum[2] $F = m \cdot a$ lorem ipsum dolor sit amet.
% Równanie bez numeru
% align oznacza wyrównanie kolejnych wierszy do '&'
% '&' służy tylko do wyrównania i nie jest renderowany
\begin{align*}
    E & = mc^2 \\
    y & = ax^2 + bx + c
\end{align*}

\lipsum[3]
% Równanie numerowane: macierze
\begin{align}
    \begin{bmatrix}
        1 & 0 & 0 \\
        0 & 2 & 0 \\
        0 & 0 & 3
    \end{bmatrix} \cdot
    \begin{bmatrix}
        4 \\
        5 \\
        6
    \end{bmatrix} =
    \begin{bmatrix}
        4  \\
        10 \\
        18
    \end{bmatrix}
\end{align}

% Cytaty dla zapełnienia bibliografii
\lipsum[4] Lorem ipsum dolor sit amet, consectetur adipiscing elit, sed do eiusmod tempor incididunt ut labore et dolore magna aliqua \cite{szczypiorski2015}, \cite{duqu2011}, \cite{shs2015}, \cite{wozniak2018}, \cite{dcp19}.

% Podrozdział pierwszego poziomu
\subsection{Critique of Pure Reason}
\kant[1]

% Tabela wielostronicowa, 4 kolumny
% Kolumny typu m{} oznaczają kolumny o stałej szerokości z zawijaniem wierszy
% Wyrównywane są domyślnie do lewej; aby ustawić inne wyrównanie,
% stosujemy \multicolumn{1} tak jak poniżej
\begin{longtable}{| c | m{0.58\linewidth} | r | m{0.1\linewidth} |}
    \caption{Tabela wielostronicowa.}
    \label{table:koszty} \\

    \hline
    % Nagłówek tabeli wyrównujemy do środka
    Lp & \multicolumn{1}{c|}{Treść} & \multicolumn{1}{c|}{Kwota} & \multicolumn{1}{m{0.1\linewidth}|}{Wariant opłaty} \\ \hline\hline \endfirsthead \endfoot
    \hline \endlastfoot

    1  & Lorem ipsum dolor sit amet, consectetur adipiscing elit, sed do eiusmod tempor incididunt ut labore et dolore magna aliqua. & 111 111,11 zł & \multicolumn{1}{c|}{WAR1} \\ \hline
    2  & Lorem ipsum dolor sit amet, consectetur adipiscing elit, sed do eiusmod tempor incididunt ut labore et dolore magna aliqua. & 22 222,22 zł & \multicolumn{1}{c|}{WAR1} \\ \hline
    3  & Lorem ipsum dolor sit amet, consectetur adipiscing elit, sed do eiusmod tempor incididunt ut labore et dolore magna aliqua. & 33 333,33 zł & \multicolumn{1}{c|}{WAR1} \\ \hline
    4  & Lorem ipsum dolor sit amet, consectetur adipiscing elit, sed do eiusmod tempor incididunt ut labore et dolore magna aliqua. & 444 444,44 zł & \multicolumn{1}{c|}{WAR1} \\ \hline
    5  & Lorem ipsum dolor sit amet, consectetur adipiscing elit, sed do eiusmod tempor incididunt ut labore et dolore magna aliqua. & 55 555,55 zł & \multicolumn{1}{c|}{WAR1} \\ \hline
    6  & Lorem ipsum dolor sit amet, consectetur adipiscing elit, sed do eiusmod tempor incididunt ut labore et dolore magna aliqua. & 66 666,66 zł & \multicolumn{1}{c|}{WAR1} \\ \hline
    7  & Lorem ipsum dolor sit amet, consectetur adipiscing elit, sed do eiusmod tempor incididunt ut labore et dolore magna aliqua. & 777 777,77 zł & \multicolumn{1}{c|}{WAR1} \\ \hline
    8  & Lorem ipsum dolor sit amet, consectetur adipiscing elit, sed do eiusmod tempor incididunt ut labore et dolore magna aliqua. & 8 888,88 zł & \multicolumn{1}{c|}{WAR1} \\ \hline
    9  & Lorem ipsum dolor sit amet, consectetur adipiscing elit, sed do eiusmod tempor incididunt ut labore et dolore magna aliqua. & 999 999,99 zł & \multicolumn{1}{c|}{WAR1} \\ \hline
    10 & Lorem ipsum dolor sit amet, consectetur adipiscing elit, sed do eiusmod tempor incididunt ut labore et dolore magna aliqua. & 111 111,11 zł & \multicolumn{1}{c|}{WAR2} \\ \hline
    11 & Lorem ipsum dolor sit amet, consectetur adipiscing elit, sed do eiusmod tempor incididunt ut labore et dolore magna aliqua. & 22 222,22 zł & \multicolumn{1}{c|}{WAR2} \\ \hline
    12 & Lorem ipsum dolor sit amet, consectetur adipiscing elit, sed do eiusmod tempor incididunt ut labore et dolore magna aliqua. & 33 333,33 zł & \multicolumn{1}{c|}{WAR2} \\ \hline
    13 & Lorem ipsum dolor sit amet, consectetur adipiscing elit, sed do eiusmod tempor incididunt ut labore et dolore magna aliqua. & 444 444,44 zł & \multicolumn{1}{c|}{WAR2} \\ \hline
    14 & Lorem ipsum dolor sit amet, consectetur adipiscing elit, sed do eiusmod tempor incididunt ut labore et dolore magna aliqua. & 55 555,55 zł & \multicolumn{1}{c|}{WAR2} \\ \hline
    15 & Lorem ipsum dolor sit amet, consectetur adipiscing elit, sed do eiusmod tempor incididunt ut labore et dolore magna aliqua. & 66 666,66 zł & \multicolumn{1}{c|}{WAR2} \\ \hline
       & \multicolumn{1}{r|}{\textbf{Suma:}} & \textbf{7 777 777,77 zł} &
\end{longtable}

\kant[2]

% Nagłówki kolejnych poziomów, dla zapełnienia spisu treści
\subsection{Caegorical Imperative} % 2.2
\subsubsection{Deontological Ethics} % 2.2.1
\kant[2]
\subsubsection{Consequentialism -- the Ideal of practical reason} % 2.2.2
\kant[3]
\subsection{G\"odel's ontological proof} % 2.3
\kant[9] Lorem ipsum dolor sit amet, consectetur adipiscing elit \cite{benzmuller2014}, \cite{goedel95}, \cite{wang97}, \cite{koons2005}.

% Twierdzenia i dowody
% Założenie
\begin{assumption} \label{ass:1}
    $ [\![ \ \phi \ ]\!] \Longrightarrow [\![ \ P(\phi); \neg P(\phi) \ ]\!]$
\end{assumption}
% Aksjomat
\begin{axiom}[Dualność] \label{axiom:1}
    $\neg P(\phi) \Leftrightarrow P(\neg \phi)$, równoważnie $P(\phi) \Leftrightarrow \neg P(\neg \phi)$
\end{axiom}
\begin{axiom}[Całkowitość] \label{axiom:2}
    $ \left( P(\phi) \wedge \forall x: \phi(x) \Rightarrow \psi(x) \right) \Rightarrow P(\psi) $
\end{axiom}
\begin{axiom}[Absolutność] \label{axiom:3}
    $ P(\phi) \Rightarrow \Box P(\phi) $
\end{axiom}
% Definicja
\begin{definition} \label{def:1}
    $ G(x) \Leftrightarrow \forall \phi: \left( P(\phi) \Rightarrow \phi(x) \right) $
\end{definition}
\begin{definition} \label{def:2}
    $ \phi \ ess \ x \Leftrightarrow \phi(x) \wedge \forall \psi \left( \psi(x) \Rightarrow \Box \forall y \left( \phi(y) \Rightarrow \psi(y) \right) \right)  $
\end{definition}
\begin{axiom} \label{axiom:4}
    P(G)
\end{axiom}
% Lemat
\begin{lemma} \label{lemma:1}
    $ P(\phi) \Rightarrow \Diamond \exists x : \phi(x) $
\end{lemma}
\begin{proof}
    Dowód pomijamy, bo jest trywialny :)
\end{proof}
\begin{lemma} \label{lemma:2}
    $ \Diamond \exists x : G(x) $
\end{lemma}
\begin{proof}
    Natychmiastowy wniosek z aksjomatu \ref{axiom:4} i lematu \ref{lemma:1}.
\end{proof}
\begin{lemma} \label{lemma:3}
    $ G(x) \Rightarrow G \ ess \ x $
\end{lemma}
\begin{proof}
    Poprzez podstawienie do definicji \ref{def:2}.
\end{proof}
\begin{definition} \label{def:3}
    $ E(x) \Leftrightarrow \forall \phi \left( \phi \ ess \ x \Rightarrow \Box\ \exists x: \phi(x) \right) $
\end{definition}
\begin{axiom} \label{axiom:5}
    P(E)
\end{axiom}
% Twierdzenie
\begin{theorem}
    $ \Box\ \exists x : G(x) $
\end{theorem}
\begin{proof}
    Na podstawie definicji \ref{def:1}, lematu \ref{lemma:3} i aksjomatu \ref{axiom:5}.
\end{proof}

\clearpage % Rozdziały zaczynamy od nowej strony.
\section{Code listings}

\lipsum[10]

% Fragment kodu źródłowego programu
% \addmargin pozwala na wcięcie kodu od lewej (tu: 8mm).
% Wcięcie służy do tego, aby numery linii nie wystawały poza lewy margines.
% Druga liczba oznacza wcięcie od prawej.
\begin{addmargin}[8mm]{0mm}
\begin{lstlisting}[
    language=HTML,
    numbers=left,
    firstnumber=1,
    caption={\emph{Hello world} w HTML},
    aboveskip=10pt
]
<html>
  <head>
    <title>Hello world!</title>
  </head>
  <body>
    Hello world!
  </body>
</html>
\end{lstlisting}
\end{addmargin}

\lipsum[11]

% Dla dłuższych numerów linii potrzebne jest większe wcięcie.
\begin{addmargin}[12mm]{0mm}
\begin{lstlisting}[
    language=C++,
    numbers=left,
    firstnumber=147,
    caption={Generowanie sekwencji Collatza w języku C++},
    aboveskip=10pt
]
class Collatz {
  private:
    unsigned current_val_;
    void update_val() {
        if( current_val_ % 2 == 0 )
            current_val_ /= 2;
        else
            current_val_ = current_val_ * 3 + 1;
    }

  public:
    explicit Collatz(unsigned initial_value) :
        current_val_(initial_value) {}
    void print_sequence() {
        unsigned i = 1;
        while( current_val_ > 1 ) {
            std::cout
                << "val " << i << " = " << current_val_
                << std::endl;
            update_val(); ++i;
        }
    }
};

int main() {
  // prints Collatz seqence, starting from 194375
  Collatz seq(194375);
  seq.print_sequence();
  return 0;
}
\end{lstlisting}
\end{addmargin}

\lipsum[12]


\cleardoublepage
\printbibliography
\clearpage

% Wykaz symboli i skrótów.
% Pamiętaj, żeby posortować symbole alfabetycznie
% we własnym zakresie. Makro \acronymlist
% generuje właściwy tytuł sekcji, w zależności od języka.
% Makro \acronym dodaje skrót/symbol do listy,
% zapewniając podstawowe formatowanie.
\acronymlist
\acronym{EiTI}{Wydział Elektroniki i Technik Informacyjnych}
\acronym{PW}{Politechnika Warszawska}
\acronym{WEIRD}{ang. \emph{Western, Educated, Industrialized, Rich and Democratic}}
\vspace{0.8cm}

%--------------------------------------
% Spisy: rysunków, tabel, załączników
%--------------------------------------
\pagestyle{plain}

\listoffigurestoc    % Spis rysunków.
\vspace{1cm}         % vertical space
\listoftablestoc     % Spis tabel.
\vspace{1cm}         % vertical space
\listofappendicestoc % Spis załączników

%-------------
% Załączniki
%-------------

% Obrazki i tabele w załącznikach nie trafiają do spisów
\captionsetup[figure]{list=no}
\captionsetup[table]{list=no}

% Załącznik 1
\clearpage
\appendix{Nazwa załącznika 1}
\lipsum[1-3]
\begin{figure}[!h]
	\centering hello
	\caption{Obrazek w załączniku.}
\end{figure}
\lipsum[4-7]

% Załącznik 2
\clearpage
\appendix{Nazwa załącznika 2}
\lipsum[1-2]
\begin{table}[!h] \centering
    \caption{Tabela w załączniku.}
    \begin{tabular} {| c | c | r |} \hline
        Kolumna 1       & Kolumna 2 & Liczba \\ \hline\hline
        cell1           & cell2     & 60     \\ \hline
        \multicolumn{2}{|r|}{Suma:} & 123,45 \\ \hline
    \end{tabular}
\end{table}
\lipsum[3-4]

\end{document}
