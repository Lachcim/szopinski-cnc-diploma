\documentclass[
    bindingoffset=5mm,
    footnoteindent=3mm,
    hyphenation=true
]{template/wut-thesis}

\facultyeiti
\EngineerThesis
\langeng

\graphicspath{{img}}
\addbibresource{bibliography.bib}

\begin{document}

\instytut{Computer Science}
\kierunek{Computer Science}
\specjalnosc{Computer Systems and Networks}
\title{
    General-purpose plotter with control software
}
\poltitle{
    Ploter ogólnego zastosowania z oprogramowaniem sterującym
}
\author{Michał Szopiński}
\album{300182}
\promotor{mgr inż. Maciej Urbański}
\date{2022}
\maketitle

% english abstract
\cleardoublepage
\abstract \lipsum[1-3]
\keywords XXX, XXX, XXX

% polish abstract
\clearpage
\secondabstract \kant[1-3]
\secondkeywords XXX, XXX, XXX

\pagestyle{plain}

\cleardoublepage
\tableofcontents

\cleardoublepage
\pagestyle{headings}

% thesis sections
\input{1-wstep}
\input{2-de-finibus}
\input{3-code-listings}

\cleardoublepage
\printbibliography
\clearpage

% Wykaz symboli i skrótów.
% Pamiętaj, żeby posortować symbole alfabetycznie
% we własnym zakresie. Makro \acronymlist
% generuje właściwy tytuł sekcji, w zależności od języka.
% Makro \acronym dodaje skrót/symbol do listy,
% zapewniając podstawowe formatowanie.
\acronymlist
\acronym{EiTI}{Wydział Elektroniki i Technik Informacyjnych}
\acronym{PW}{Politechnika Warszawska}
\acronym{WEIRD}{ang. \emph{Western, Educated, Industrialized, Rich and Democratic}}
\vspace{0.8cm}

%--------------------------------------
% Spisy: rysunków, tabel, załączników
%--------------------------------------
\pagestyle{plain}

\listoffigurestoc    % Spis rysunków.
\vspace{1cm}         % vertical space
\listoftablestoc     % Spis tabel.
\vspace{1cm}         % vertical space
\listofappendicestoc % Spis załączników

%-------------
% Załączniki
%-------------

% Obrazki i tabele w załącznikach nie trafiają do spisów
\captionsetup[figure]{list=no}
\captionsetup[table]{list=no}

% Załącznik 1
\clearpage
\appendix{Nazwa załącznika 1}
\lipsum[1-3]
\begin{figure}[!h]
	\centering hello
	\caption{Obrazek w załączniku.}
\end{figure}
\lipsum[4-7]

% Załącznik 2
\clearpage
\appendix{Nazwa załącznika 2}
\lipsum[1-2]
\begin{table}[!h] \centering
    \caption{Tabela w załączniku.}
    \begin{tabular} {| c | c | r |} \hline
        Kolumna 1       & Kolumna 2 & Liczba \\ \hline\hline
        cell1           & cell2     & 60     \\ \hline
        \multicolumn{2}{|r|}{Suma:} & 123,45 \\ \hline
    \end{tabular}
\end{table}
\lipsum[3-4]

\end{document}
