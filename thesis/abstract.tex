\abstract
The goal of this thesis was to construct a general-purpose CNC machine capable
of applying a variety of drawing tools to a flat surface. The machine
must be easy to set up and use by non-tech-savvy users. This is to be
achieved by designing a GUI application that interfaces between users and the
machine, guiding them through common workflows while keeping the technicalities
to a minimum. The thesis describes the industry conventions on designing and
programming CNC machines, then proceeds to describe the designed solution: the
hardware, the firmware, and the control software. The machine was successfully
constructed and tested in practical use-case scenarios. The results of the tests
are presented and discussed in the thesis.

\keywords CNC, microprocessor systems, bare-metal programming, GUI applications

\clearpage
\secondabstract
Tematem pracy dyplomowej jest konstrukcja urządzenia CNC ogólnego zastosowania,
zdolnego do posługiwania się różnymi narzędziami kreślarskimi na płaskich
powierzchniach. Urządzenie to musi być łatwe w instalacji i obsłudze przez
użytkowników nietechnicznych. Wymóg ten należy spełnić poprzez zaprojektowanie
aplikacji graficznej, stanowiącej interfejs między użytkownikiem a urządzeniem,
która przeprowadzi go przez realizację powszechnych zadań, jednocześnie
ograniczając zagadnienia techniczne do minimum. Praca opisuje przyjęte standardy
projektowania i programowania urządzeń CNC, a następnie przedstawia opracowane
rozwiązanie: układ elektroniczny (hardware), oprogramowanie układowe (firmware)
oraz oprogramowanie sterujące (software). Urządzenie zostało pomyślnie
skonstruowane i przetestowane pod kątem zdolności do zastosowania w praktyce.
Wyniki testów zostały przedstawione i omówione w niniejszej pracy.

\secondkeywords CNC, systemy mikroprocesorowe, programowanie niskopoziomowe,
aplikacje graficzne
