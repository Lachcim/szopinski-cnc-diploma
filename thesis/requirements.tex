\clearpage
\section{CNC machine design}

\subsection{Project requirements}

The goal of this project is to construct a general-purpose CNC machine capable
of applying a variety of drawing tools to a flat surface. This involves
the capability to freely move a tool in three-dimensional space, with a large
range of motion in the horizontal plane and a relatively restricted range of
motion in the vertical axis. It must be compatible with common workpiece
formats, including A4 paper. It must have the ability to use interchangeable
tools of different shapes and sizes.

Because the machine does not face much physical resistance during normal
operation, there is no need to monitor the tool position - it may be assumed
that it is in accordance with the executed instructions. As such, an open-loop
control scheme may be implemented. Similarly, because the machine is not used
to remove material from the workpiece, it may operate at a pre-defined constant
feed rate, low enough for open-loop control.

To achieve partial compatibility with other solutions available on the market,
it must use G-code as the input instruction format, and it must implement a
large enough subset of G-code to implement basic functionality. This includes:
\begin{enumerate}
    \item Three types of motion: rapid, linear and arc
    \item Absolute and incremental coordinate mode
    \item Metric and Imperial units of measurement
    \item Commands and syntax related to feed rate, even though the
    functionality itself is not implemented
\end{enumerate}

The machine uses a serial port to communicate with its controlling computer.
Because modern personal computers do not come with a physical RS-232 serial
port, the connection must be established over USB, using generic drivers to
create a virtual serial port.

To enhance user experience, the machine must provide feedback in response to
input. The feedback includes:
\begin{enumerate}
    \item The current physical coordinates of the work tool, sent periodically
    \item A message sent whenever a commands starts or finishes executing
    \item An error code describing any syntax or semantic errors in the command
    \item An interpretation of the currently executing command, including:
        \begin{itemize}
            \item The type of movement it triggered
            \item The source and destination of the movement
            \item The center of the arc, in the case of arcs
        \end{itemize}
\end{enumerate}

Table~\ref{projectRequirements} contains a summary of the requirements
of the CNC machine.

\begin{table}[ht]
    \begin{center}
        \begin{tabular}{ |l|c|c| }
            \hline
            Parameter & Value & Unit \\
            \hline
            Axes of motion & 3 & - \\
            Controller type & Open-loop & - \\
            Input method & G-code over serial port & - \\
            Connection interface & USB & - \\
            Supported motion types & Rapid, linear, arc & - \\
            X axis travel & 200 & mm \\
            Y axis travel & 200 & mm \\
            Z axis travel & 5 & mm \\
            Max tool diameter & 18 & mm \\
            Max workpiece width & 297 & mm \\
            \hline
        \end{tabular}
        \caption{Project requirements of the CNC machine}
        \label{projectRequirements}
    \end{center}
\end{table}

The machine must be user-friendly. It must come with control software which
guides the user through common use-case scenarios. These scenarios are:
\begin{enumerate}
    \item Batch execution, where the user specifies a file containing
    instructions to be executed in sequence
    \item Command prompt, where the user may type commands by hand and observe
    their effects
    \item Bitmap tracing, where the user submits a raster image to be converted
    into a vector approximation, which can then be drawn by the machine
\end{enumerate}

The software must allow the user to select which serial port to use. It detects
the presence of the CNC machine on the other end informs the user if a
connection problem occurs. During program execution, the user may view each
command, see its execution status, its associated error message, and a graphical
representation of the motion it initiated.
