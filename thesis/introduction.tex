\clearpage
\section{Introduction}

\subsection{CNC machine design}

Computer numerical control (CNC) is a technique which utilizes a computer to
automate the control of a machining tool. By modern definitions, ``machining''
refers to a process in which a material is manipulated to obtain a desired
shape, either through subtractive or additive manufacturing. Machining tools
provide a means to guide a cutting tool along a strictly controlled path
relative to the workpiece, free of human error.

A machining tool with an extensive range of motion in the horizontal plane, but
a limited range of motion in the vertical axis, is commonly called a plotter.
Before the advent of the inkjet printer, these devices were used in engineering
settings to plot functions and draw line art on paper. In modern contexts,
plotters are employed to create inscriptions on rigid or otherwise
unconventional surfaces, where printers may not be used.

In the recent years, CNC machines have been rapidly growing in popularity as
household devices. This is especially true of one such machine, the 3D printer,
which moves a nozzle in three-dimensional space to create a shape out of molten
plastic. As consumers become familiar with CNC technologies, computer-controlled
plotters, drills and lathes may soon become standard workshop equipment.

The concept of numerical control of machining tools is not new; the first
solutions involving punched cards date back to the 1950s. Since that time,
various manufacturers have implemented their own standards for communicating
with machines, the most prominent of which is a data format named G-code.
Originally developed by Gerber Scientific, the format was adopted by many
companies and has become a de-facto industry standard.

In its original form, G-code lacked the ability to read, manipulate and store
data, which made it more akin to a vector graphics description format than to a
programming language. Basic G-code instructions modify the state of the machine
and instruct it to move to a specified position. The machine state impacts the
interpretation of commands and the interpolation between requested positions.

The basic set of instructions enables the operator to (including but not
limited to):
\begin{itemize}
    \item Move in a straight line, draw an arc, or move the tool rapidly
    \item Adjust the movement speed
    \item Switch between absolute and relative positioning
    \item Change the origin of the coordinate system
    \item Switch between metric and US customary units
\end{itemize}
Various manufacturers, however, have opted to implement their own versions of
G-code. This includes differences in instruction encoding, interpretation, and
parsing. Only a small subset of instructions and syntaxes may be expected to
work on all machines universally.

Due to G-code's origin as a vector data format rather than a communication
protocol, there is also no standarized way to obtain any sort of feedback from
the machine to its controlling computer. To establish two-way communication
with the machine, manufacturers employ proprietary solutions.

\subsection{GUI application design}

Modern developments in user interface design show a tendency to step away from
native user controls in favor of web technologies. Desktop and mobile
applications present their interface as an embedded browser window where
controls are implemented using HTML, CSS and JavaScript. This improves user
experience by presenting the user with a modern and familiar UI.

A commonly used software framework for creating this kind of applications is
Electron. It consists of three crucial components:
\begin{itemize}
    \item A standalone JavaScript engine, Node.js. This enables JavaScript to
    be executed outside of a browser context, thanks to which the entire
    application may be written in JavaScript.
    \item JavaScript bindings for native OS functionality, such as interacting
    with the window system or capturing the screen contents. Node.js, being a
    web server by design, lacks this functionality.
    \item A browser context in which the UI code is executed and rendered.
\end{itemize}

When building complex UI applications, it is important to maintain and follow
a strict set of rules regarding data flow and update logic. Between managing
the global program state, handling user interactions and rendering interface
updates, it may be difficult to write readable and maintainable code. A popular
solution to this problem is the React framework, which facilitates the
development of UI applications by introducing the following design principles:
\begin{itemize}
    \item The UI is split into components, each of which is only responsible
    for rendering a small part of the application. Components maintain the
    logical state and handle the side effects of whatever abstraction they
    represent.
    \item Components follow a declarative convention where their rendered
    output is determined solely by their state. This removes the need for
    update logic.
    \item Components can only directly influence components directly contained
    within them. This creates an unidirectional data flow that is easy to manage
    and maintain.
\end{itemize}

\subsection{About this thesis}

The goal of this thesis was to design and to construct a CNC plotter together
with the software needed to control it. The machine implements a subset of the
G-code instruction set that is suitable for general-purpose plotting operations
on flat surfaces. The software is aimed at the consumer market and as such, it
must be easy to set up and use. It is a GUI application which communicates with
the machine over a wired connection.

As part of the project, the following was done:
\begin{enumerate}
    \item A mechanical system capable of moving a drawing instrument in
    three-dimensional space was prepared. The system is a combination of
    pre-made aluminum components and custom-made 3D-printed connector parts.
    \item A microcontroller system was designed to drive stepper motors and to
    communicate with control software. It utilizes ready-made printed circuit
    boards containing driver ICs and the necessary discrete components. A
    custom-made PCB was prepared to house the microcontroller chip together
    with its peripherals.
    \item A firmware program was created to execute CNC instructions and to send
    feedback to control software. It features original code only.
    \item A GUI application was developed to act as an interface between the
    user and the machine. Several common software frameworks were used to
    implement the generic parts of the program.
\end{enumerate}

This thesis is divided into several sections discribing the hardware, the
firmware and the software individually. Each section begins with an analysis
of the relevant project requirements, then goes on to provide a general
overview of the accepted solution, and in the end gives a detailed description
of each major component of the implementation. The final section of this
document contains an analysis of the finished project and drawn conclusions.

Attached to this document is the source code for all described software, the
schematics of the designed microcontroller system and its PCB, and technical
drawings showing the mechanical system.
