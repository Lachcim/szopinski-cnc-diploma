\clearpage
\section{Introduction}

Computer numerical control (CNC) is a technique which utilizes a computer to
automate the control of a machining tool. By modern definitions, ``machining''
refers to a process in which a material is manipulated to obtain a desired
shape, either through subtractive or additive manufacturing. Machining tools
provide a means to guide a cutting tool along a strictly controlled path
relative to the workpiece, free of human error.

A machining tool with an extensive range of motion in the horizontal plane, but
a limited range of motion in the vertical axis, is commonly called a plotter.
Before the advent of the inkjet printer, these devices were used in engineering
settings to plot functions and draw line art on paper. In modern contexts,
plotters are employed to create inscriptions on rigid or otherwise
unconventional surfaces, where printers may not be used.

The goal of this thesis was to design and to construct a CNC plotter together
with the software needed to control it. The machine implements a subset of the
G-code instruction set that is suitable for general-purpose plotting operations
on flat surfaces. The software is aimed at the consumer market and as such, it
must be easy to set up and use. It is a GUI application which communicates with
the machine over a wired connection.

As part of the project, the following was done:
\begin{enumerate}
    \item A mechanical system capable of moving a drawing instrument in
    three-dimensional space was prepared. The system is a combination of
    pre-made aluminum components and custom-made 3D-printed connector parts.
    \item A microcontroller system was designed to drive stepper motors and to
    communicate with control software. It utilizes ready-made printed circuit
    boards containing driver ICs and the necessary discrete components. A
    custom-made PCB was prepared to house the microcontroller chip together
    with its peripherals.
    \item A firmware program was created to execute CNC instructions and to send
    feedback to control software. It features original code only.
    \item A GUI application was developed to act as an interface between the
    user and the machine. Several common software frameworks were used to
    implement the generic parts of the program.
\end{enumerate}

The thesis is divided into several sections. The section \textit{Theoretical
background} provides a brief introduction to the theory behind designing CNC
machines and GUI applications. The next section, \textit{Analysis of
market-available solutions}, aims to set a frame of reference for the project by
looking at existing CNC solutions. The section \textit{CNC machine design}
begins by formulating the set of project requirements, after which a detailed
description of the solution is given. In \textit{Tests and verification}, the
solution is evaluated to see if it meets the project requirements. Finally, a
summary is provided, where conclusions are drawn.

Attached to this thesis is a storage medium with relevant materials. Its
contents are listed in the list of appendices at the end of this document.
